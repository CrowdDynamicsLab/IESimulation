\documentclass{article}
\usepackage[utf8]{inputenc}
\usepackage[left=2cm, right=2cm, top=2cm]{geometry}
\usepackage{graphicx}
\usepackage{amsmath}
\usepackage{amssymb}
\usepackage{mathrsfs}
\usepackage[ruled,vlined]{algorithm2e}
\usepackage{float}

\begin{document}

The information elicitation game is modeled on a network $G$ with vertices $V$ representing people and edges $E$ representing relationships between people.
Each vertex $v \in V$ is given an initial utility $v_u$ from $0$ to $1$
Each vertex $v \in V$ is allocated an initial amount of resources, $v_r$.

In each turn of the game we iterate over every vertex with resources greater than $0$.
Each vertex selects a neighbor with resources greater than $0$ by a strategy $S$.
$S(v)$ represents the neighbor selected by vertex $v$ using strategy $S$.
After selection one resource is subtracted from that vertex and the selected neighbor.
Then with probability $p$ the vertex and its neighbor exchange information.
Both the vertex and its neighbor switch to whichever provider gives the highest utility.

The game ends when all resources have been expended.

In other words,

\begin{algorithm}[H]
\SetAlgoLined
 let $G = (V, E)$\;
 let $S$ be the strategy used\;
 \While{$v_r > 0 \ \forall v \in V$} {
 	\For{$v \in V$} {
 		\If{$v_r = 0$} {
 			continue\;
 		}
 		
 		\If{$\sum_{w \in N(v)} w_r = 0$} {
 			let $v_r = 0$\;
 			continue\;
 		}
 		
 		let $w = S(v)$\;
 		$v_r = v_r - 1$\;
 		$w_r = w_r - 1$\;
 		$v_u = w_u = \max\{v_u, w_u\}$\;
 	}
 }
 \caption{Information elicitation}
\end{algorithm}

As this is an iterative process we may denote the utility of a given vertex $v$ at a turn $t$ as $v_r^t$.
A basic measure of the efficiency of a network $G$ or strategy $S$ can be represented as $\frac{\sum_{v \in V} v_r^T}{|V|}$ where $T$ is the final iteration.

\end{document}