\documentclass{article}
\usepackage[utf8]{inputenc}
\usepackage[left=2cm, right=2cm, top=2cm]{geometry}
\usepackage{graphicx}
\usepackage{amsmath}
\usepackage{amssymb}
\usepackage{mathrsfs}
\usepackage[ruled,vlined]{algorithm2e}
\usepackage{float}

\begin{document}

\section{Deterministic existance of edges}

Given a set of vertices $V$ which have two attributes, $A$ and $B$, we will have an edge between vertices $u, v \in V$ having opposing advantages in $A$ and $B$.
That is $uv \in E$ the set of edges $\iff u_A > v_A \text{ and } u_B < v_B$
or $u_A < v_A \text{ and } u_B > v_B$.
Suppose the attribute values are independently selected from a distribution of the integers from $1$ to $k$ inclusive.
Then the probability of an edge forming is $2 * (\frac{1 - \Pr[u_A = v_A]}{2} * \frac{1 - \Pr[u_B = v_B]}{2}) = \frac{(1 - \frac1k)^2}{2}$.
In the case where $k = 10$ this probability is $0.405$.

This gives rise to an Erdos-Renyi graph with edge probability $\frac{(1 - \frac1k)^2}{2}$.
The sharp boundary for connectivity in an Erdos-Renyi is an edge probability of $\frac{\ln n}{n}$ where $n = |V|$.
Thus for a given $k$ we must select an $n$ s.t. $\frac{(1 - \frac1k)^2}{2} > \frac{\ln n}{n}$.
Because $\frac{\ln n}{n}$ is monotonically decreasing for all $n > 1$ and $n = 1, 2$ fulfill this criteria all values of $n$ will meet the Erdos-Renyi threshold for a connected graph.
However because there is a $\frac{2}{k^2}$ probability of a vertex having either both the highest or the lowest attribute values in expectation there will be $\frac{2n}{k^2}$ vertices that have no neighbors and so depending on the value of $n$ the graph may have a high probability of being disconnected.

When instead of $k$ options we select a real number from $0$ to $1$ we instead have an edge probability of $\frac12$. Similarly any value of $n$ will give a connected component.
Here the probability that a vertex has the max or min attribute values is $2*(\frac12)^{n-1}$

By simulation this tends to give shorter average path lengths than a Kleinberg grid which is expected given the known expected path lengths in Erdos-Renyi and Kleinberg grid graphs.
% https://arxiv.org/abs/cond-mat/0407098 gives ER APL
% https://www.researchgate.net/publication/221344073_Analyzing_Kleinberg's_and_other_small-world_Models references KG APL

\section{Probabilistic existance of edges}

Consider a formulation where the attributes determine the probability of an edge existing.
We will introduce a new parameter $C$ with defines the initial edge probability.
Thus we start with an Erdos-Renyi graph having edge probability $C$.

Next as before we discard all edges where the incident vertices do not have attribute differentials.
Then we keep edges with probability $-1 * (u_A - v_A) * (u_B - v_B)$. % isn't this 0?

\end{document}